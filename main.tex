\documentclass[12pt,a4paper,fleqn]{article}

%%% REPORT FOR SPIN WAVES EXCITATION BY ACOUSTIC PULSE

\usepackage{amsmath}               %AMSMath tools for flexible equations
\usepackage[english]{babel}
\usepackage{graphicx}
\usepackage{cite}
\usepackage[margin=1.5cm]{geometry}
\usepackage{subfig} % remplace subfigure
%\usepackage[utf8x]{inputenc} 
\usepackage[latin1]{inputenc} 
\usepackage[T1]{fontenc} 
\usepackage{multirow}
\usepackage[section]{placeins}
\usepackage{subfig} % remplace subfigure
% \usepackage{pdfpages} % pour inserer des pdfs
\usepackage{setspace}
\onehalfspacing
%\doublespacing


\renewcommand{\thefootnote}{\fnsymbol{footnote}}
\DeclareTextSymbol{\degre}{T1}{6}

\begin{document}
	
	
\begin{center}
	{\bf\Large Report~\MakeUppercase{\romannumeral 3}: Ultrafast Lattice Deformation - Determination of the thermal conductivity and the thermal resistance}\\
	%{\bf\Large Preliminary investigations.}\\
	\vspace{0.4cm}
	{\large Valentin \textsc{Besse}\footnote{Author of the report}, Peter \textsc{Gaal}, Mathias \textsc{Sanders}, Vasily \textsc{Temnov}}\\
	\vspace{0.6cm}
	{\large January 31, 2017}
	%{\large \today}
\end{center}
\vspace{0.1cm}

\begin{abstract}
	{\noindent In this report, we present the surface deformation curve as a function of time.
	This curve correspond to the experimental data once we remove the surface acoustic wave and background.
	We calculate the thermal diffusion in the bi-layer sample as a function of time for different values of the first layer's thermal conductivity and the interfacial thermal resistance.
	We determined the values of these two parameters that correspond to the curve which reproduce the experimental curve decay.
	This estimation is achieved visually.}
\end{abstract}

\section{Version}

\begin{itemize}
	\item V. 0.1: January 24, 2017. Creation of the file
	\item V. 0.2: January 30, 2017.
\end{itemize}

\end{document}